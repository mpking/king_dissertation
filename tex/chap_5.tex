%!TEX root = /Users/mpking/Dropbox/writing/king_dissertation/king_dissertation_master.tex
\chapter{Summary and Conclusions} % (fold)
\label{cha:summary_and_conclusions}
Evidence of single electron-induced SEU in 28 and 45~nm CMOS SRAMs is presented. 
Energetic electrons are generated by exposure of the SRAMs to an X-ray source and an aluminum attenuator. 

The experimental SEU cross-sections depend exponentially on applied bias, consistent with previous experimental results obtained with muons and low-energy protons. 
No errors were observed in functionality and parametric testing before and after irradiation of all test chips under all applied bias conditions. 
This demonstrates that test chips remained stable during X-ray irradiation. 
Thus, errors are not due to ``weak bits'' or photocurrents resulting from the collective energy deposition of the X-rays. 
Instead, experimental results and analysis strongly suggest that the observed errors are the result of single energetic electron scattering events within SRAM cells. 

The event rate of electron-induced SEU is low under nominal bias conditions at geosynchronous orbits for the devices that were evaluated. 
Similarly, electron-induced SEUs in the Jovian environment are predicted to be rare events occuring only slightly more frequently than in the near-Earth environment.
However, operating microelectronic systems in power-saving or quiescent mode would significantly increase the likelihood of electron-induced SEUs contributing to anomalous behavior in the onboard electronics systems while within the Jovian electron belts.

Moreover, electron-induced upsets have only been observed to occur at measurable rates under reduced bias conditions for SRAMs fabricated in present-generation technology nodes. 
This suggests that the overall contribution of energetic electrons to error rates is small in current-generation technology.
The conclusion being that electronics designed to operate with ultra-low power likely will exhibit higher relative sensitivity to energetic electron induced upsets. 
This represents an additional design concern for both space and terrestrial environments, to avoid unexpectedly high upset/error rates from lightly ionizing particles.

Monte Carlo radiation transport simulation results indicate that in technology nodes where less than 0.5~fC of charge result in circuit-level effects, $\delta$-rays generated in heavy-ion irradiation may contribute to the single- and multiple-bit error rate.
Error rates of $\delta$-ray and electron-induced upsets for SRAMs in ground-based parts qualification testing and the space radiation environment are likely dominated by extreme energy deposition events.
A comparison of MRED with the Katz model demonstrates average track structure models alone are inadequate in capturing the SEU response of small sensitive geometries with low critical charge that are susceptible to electron/$\delta$-ray effects.
The probability of $\delta$-ray related effects exhibits a strong dependence on the incident ion species, energy, and LET.
Additionally, the probability of $\delta$-ray induced effects exhibits a strong dependence on radial distance from the incident ion trajectory.
These results have strong implications for ground-based parts qualification testing and space radiation environments, where varying incident ion energy and LET result in differing contributions from $\delta$-rays to device and circuit level effects.
% chapter summary_and_conclusions (end)