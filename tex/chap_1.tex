%!TEX root = /Users/mpking/Dropbox/writing/king_dissertation/king_dissertation_master.tex
\chapter{Introduction} % (fold)
\label{cha:introduction}
The ability to store and read information is critical for reliable system operations in modern electronics. 
Information is stored in densely packed arrays of devices and circuits whose purpose is to maintain mission critical instructions, record data, and return that information for further computation or analysis.
Circuit- and device-level memories exist in the form of volatile and non-volatile elements; the static random access memory (SRAM) being a circuit element volatile memory, which is widely used in many applications.

SRAMs, and related memories, are renown for fast read and write times, small areal density, exhibit a non-destructive read operation, and do not require periodic refreshing of data since information persists while the memory is powered.
The SRAM represents a stable memory that has become an essential circuit-level cell because it has the ability to rapidly access and store information, making it ideal for low-power and mobile applications as well as microprocessor and system level cache.
Radiation sensitivity of circuit-level memory is an important consideration when evaluating reliability concerns of modern technology in a variety of hazardous environments including military, space, nuclear, and the terrestrial level. 
Single-event upset (SEU) is an example of the sensitivity of modern microelectronics to ionizing radiation. SEUs are defined as the erroneous change of state of a semiconductor memory, such as an SRAM, resulting from energy deposition by an ionizing particle that results in charge generation within a sensitive region of the microelectronic element.

The semiconductor industry continues to scale complementary metal-oxide-semiconductor (CMOS) technologies to smaller feature sizes with reduced operating voltages in pursuit of performance and density goals.
Continuing decreases in device dimension and operating voltage reduce the critical charge required to produce a SEU, significantly affecting the reliability of modern technologies in space and terrestrial environments. 
Decreasing critical charge has led to the emergence of SEUs induced by lightly ionizing particles, such as low-energy protons and muons \cite{Rodbell:2007vl, Sierawski:2010cj}. 
Traditionally, the primary radiation effects caused by energetic electrons in the trapped radiation environments of Earth and Jupiter were considered to be total-ionizing dose (TID), displacement damage (DD), and spacecraft charging (or electrostatic discharge (ESD)) \cite{Bourdarie:kj, Xapsos:2013cu}. 
However, in recent years interest has emerged regarding effects related to the spatial distribution of charge produced by lightly ionizing particles, including high-energy secondary electrons \cite{Weller:2003je, Raine:2010cna, Raine:gk, Raine:2012gi, King:2010cu, King:2012cb, Barak:2012im}. 
These secondary electrons, also called $\delta$-rays, lose their kinetic energy through ionization, producing electron-hole (\emph{e-h}) pairs that may cause SEUs. 
Various studies \cite{Weller:2003je, Raine:2010cna, Raine:gk, Raine:2012gi, King:2010cu, King:2012cb, Barak:2012im} have obtained conflicting results in attempts to quantify, primarily via modeling and simulation, the contributions of energetic electrons to the overall upset rate in memories fabricated at advanced technology nodes. 
Despite extensive efforts, lack of experimental data has left the role of energetic electrons in the SEU response of modern SRAMs an open question.

In this dissertation, a low-energy X-ray source is used to generate energetic, nearly ballistic electrons to evaluate the susceptibility of CMOS SRAMs fabricated in the 28~nm and 45~nm technology nodes to electron-induced SEUs. 
Throughout this dissertation, ``electron-induced SEUs'' refer to events in which the initiating particle is a high-energy electron ($\delta$-ray); the eventual upsets are produced by thermalized electron-hole pairs generated as the $\delta$-rays lose their energy through ionization. 
Upsets are observed within 10\% of nominal supply voltage for the 28~nm technology node. 
That these memory upsets are indeed electron-induced SEUs is supported by Monte Carlo radiation transport simulations, which show that single energetic electrons deposit sufficient ionizing energy to generate charge in the sensitive volume of the device that is well in excess of estimated critical charge values. 
The relative importance of electron-induced SEUs is compared to other physical processes, such as direct ionization from low-energy protons \cite{Rodbell:2007vl, Heidel:2008vf, Heidel:2009vx, Sierawski:2009ka} and muon-induced upsets \cite{Sierawski:2010cj, Sierawski:2011bn} in determining error rates of selected SRAMS fabricated in 28~nm and 45~nm technology generations. 
The impact of electron-induced SEU on scaling of feature size and voltage in modern CMOS processes, ultra-low power applications, and error rates in the space radiation environment is discussed in detail.

The organization of this dissertation is as follows. 
Chapter~\ref{chap:background} presents relevant background material including a review of SRAM topology, operation, stability, discussion of relevant topics of radiation effects in SRAMs, a review of the space radiation environment, and a summary of past work on ionizing particle track structure. 
Chapter~\ref{cha:experimental_investigation_of_electron_induced_seus} will present the experimental setup and methods used in this work, show and discuss experimental results of SEUs observed during X-ray irradiation of 28~nm and 45~nm bulk silicon SRAMs, and compare electron-induced SEUs to low-energy proton and muon data.
Chapter~\ref{ch:simulation_of_electron_induced_seus} presents supporting simulation results for the X-ray that show good agreement with experimental results. Error rates are estimated using simulation techniques, the consequences and importance of these results are discussed for the space radiation environment.
Analysis of the contribution of $\delta$-rays generated in heavy-ion irradiation to single- and multiple-bit upset rates is also discussed.
Finally, Chapter~\ref{cha:summary_and_conclusions} presents conclusions and discusses the significance of these results for modern technology nodes.
% chapter introduction (end)
